%\documentclass[12pt]{article}
\documentclass[12pt]{revtex4}
\usepackage{amsmath}
\usepackage{amsfonts}
\usepackage{amsbsy}
\usepackage{lscape} 
\usepackage{color}
\usepackage{graphicx,epsfig}
\usepackage[english]{babel}
\usepackage{latexsym}
\usepackage{amssymb}
%\usepackage{sparticles} 	%Package for displaying sparticle names. 
%\usepackage{feynmf}		%Package for feynman diagrams. 
%% slashed symbols
\newcommand{\slashed}[1]{\hbox{{$#1$}\llap{$/$}}}
\newcommand{\sslashed}[1]{\hbox{{$#1$}\llap{$/\,$}}}

\begin{document}


%%
%% The title page
%% 
\begin{titlepage}
\renewcommand{\thefootnote}{\fnsymbol{footnote}}

\begin{center}
%%
%% Title itself
\vspace{0.5cm}

\large {\bf Signatures of CPT-violating Electric Dipole Moment from the 
	Lorentz Violating Standard Model}\\[3mm]
  
\vspace*{0.5cm}
\normalsize
%\title{Lorentz Violating Supersymmetric Quantum Electrodynamics}
{\bf Pavel A. Bolokhov}$^{1,2}$ 
\ and
{\bf Maxim Pospelov}$^{1,3,4}$%\footnote{pospelov@uvic.ca}

\vspace*{0.5cm}
$^{1}$ {\it Department of Physics and Astronomy,
University of Victoria, Victoria,\\ BC, V8P 1A1, Canada}\\
$^{2}${\it Theoretical Physics Department, St.Petersburg State University, Ulyanovskaya 1,\\
Peterhof, ~St.Petersburg, 198504, Russia}\\
$^{3}$ {\it Perimeter Institute, 31 Caroline Street North,
Waterloo, ON,  N2J 2W9,
Canada}\\
$^{4}$ {\it Department of Physics,
 University of Guelph,
 Guelph, ON,  N1G 2W1, Canada}
 \end{center}

\centerline{\large\bf Abstract}
Generic Lorentz Violating parameters of mass dimension 5 are introduced 
in the matter sector of the Standard Model.
We discuss the low-energy implications of these terms and find a CPT-violating
contribution to the Electric dipole moments in the matter sector. 
To the leading order, the effect vanishes for electrons, manifesting itself
only in the quark sector, and thus, inducing the non-zero nucleon EDM. 
Based on known constraints on EDM of the neutron [and of diamagnetic atoms], we
put limits on a parameter subspace of the Lorentz Violating Standard Model.
\end{titlepage}


%%%%%%%%%%%%%%%%%%%%%%%%%%%%%%%%%%%%%%%%%%%%%%
%%%%
%%%               Introduction
%%%%
%%%%%%%%%%%%%%%%%%%%%%%%%%%%%%%%%%%%%%%%%%%%%%
\section{Introduction}
\label{Intro}

	Recently there have been many conjectures proposed 
\cite{MP:,Kost1}
	that the New Physics --- which is being so intensively pursued both 
	theoretically and phenomenologically --- can shake one of the 
	foremost pilasters of the Quantum Field Theory, the Lorentz symmetry.
	New Physics could endow our present spacetime with some intrinsic 
	constant backgrounds, which would cause anisotropy of the four
	dimensional Minkowski space. 
	Quite commonly this anisotropy is motivated by the existence 
	of the preferred frame of the cosmic microwave background.
	Even if this anisotropy exists in any form, it has already been severely
	limited by both direct and indirect experimental tests and observations
\cite{Jacobsonreview}.
	This might cause some conceptual problems to the theories which
	predict violation of the Lorentz symmetry.
	Indeed, the mechanisms of breaking of the Lorentz symmetry, dynamical
	or hard, are not known, as is the New Physics.
	
	Studies of violation of the Lorentz symmetry, however, do not need to
	involve detailed knowledge of the dynamics at UV scales. 
	We can perceive our world of today as perfectly described by the 
	Standard Model, with, probably, some tiny inflections caused by
	the anisotropic backgrounds being sought. 
	Whatever happens at the Planck scale, if it breaks Lorentz invariance,
	it has to be encoded in various effective interactions with explicit
	Lorentz symmetry breaking.
	Using the effective field theory approach, one can admit temporarily
	Lorentz-breaking structures as perturbations to the Standard Model
	and then predict detectable phenomenological deviations. 

	Therefore, signatures of the New Physics can be sought as new LV
	interactions, which can also break discrete symmetries (CPT).
	In this respect, popular candidates for new interactions are 
	terms of mass dimension five. 
	Their exceeding the canonical lagrangian mass dimension by one
	is assumed to be compensated by a factor of $ 1/\Lambda_{UV} $ ---
	the natural scale pertinent to the New Physics --- which yields
	a good deal of suppression, and could provide a explanation
	why these new interactions have not been detected.

	In this work we aim to oppose this quest for Lorentz symmetry
	violation to one of the finest experimental endeavors, precision
	of which reaches inconceivably high levels --- to the searches
	for permanent Electric Dipole Moments (EDM) 
\cite{Pospelov:2005pr}.
	It turns out, that among other experimental methods which can
	provide some quantitative answers about Lorentz violations, this
	method also beats high stakes due to its incredible sensitivity.
	It is possible to join the two problems together by observing
	that Lorentz Violation can in principle cause various hardonic
	and atomic systems to have a permanent EDM --- subject to detection
	by the EDM searches. 

	Being more specific, one can remark that the proposed LV by means
	of dimension five operators generically breaks CPT invariance.
	The intrinsic EDMs of particles will therefore also inherit this
	property which distinguishes them from regular CPT invariant dipole
	moments. 

	The goal of the present paper is to relate the searches for the
	Electric Dipole Moments to the study of dimension five LV interactions	
	treated as perturbation to the Standard Model, which are capable of
	inducing CPT-nonconserving particle EDMs.

	Due to our ignorance of the UV physics, we generically have to 
	account for all possible mass dimension five LV structures 
	admitted by gauge invariance of the Standard Model.
	All kinds of interactions must be constructed involving all
	the field content of the Standard Model.
	However, heading for the EDM effects in the low energy dynamics, 
	at tree level, one can limit himself only with the Matter part
	of the Standard Model. 

	We begin by systematically classifying dimension five LV operators
	for a generic U(1) theory, study the low-energy effects of the 
	resulting theory with respect to induction of EDM, and then generalize
	the results to the case of the Standard Model. 
	At the end, predictions on quark LV parameters of the LV Standard 
	Model are made. 

%%%%%%%%%%%%%%%%%%%%%%%%%%%%%%%%%%%%%%%%%%%%%%
%%%%
%%%               Generic Lorentz Violation
%%%%
%%%%%%%%%%%%%%%%%%%%%%%%%%%%%%%%%%%%%%%%%%%%%%
\section{Generic Lorentz Violation in the Matter Sector}

	First we study a simple case of a U(1) model with dimension five
	LV operators --- the so-called Myers-Pospelov Electrodynamics 
\cite{MP:}
	---
	with a distinction that in our case we find \emph{all} admissible
	vector and tensor LV structures of mass dimension five.

	Derivation of the LV operators generically means merely enumerating all possible
	structures which are gauge invariant and at the same time have a number of
	hanging Lorentz indices. 
	A more systematic approach is to list all gauge-invariant field bilinears, 
	trilinears or quadri-linears and introduce a (corresponding) number of 
	gauge derivatives into them in order to have external indices. 
	The reason for that is that the only source which can bring in an additional
	index into a structure is the vector field (if it is present) of the theory.
	In our case, such a vector field is the photon, and gauge invariance
	calls us to include it in the form of the gauge derivative 
	$ \mathcal{D}_\mu $.
	This will generate a big list of different structures, but yet there is no
	complete warranty that one has not missed some peculiar forms --- some cross
	checks have to be done. 
	The obtained structures have then to undergo some further stages:
	they must be expanded into irreducible components of Lorentz group 
	representations.
	This breaks them into a number of classes, e.g.:
\[
	T^{\mu\nu} ~=~ \frac{1}{2}T^{[\mu\nu]} ~+~
	\frac{1}{2} 
	\left( T^{(\mu\nu)} - \frac{1}{2}g^{\mu\nu}\,
			T^\lambda_{\phantom{\lambda}\lambda} \right) ~+~
	\frac{1}{4} g^{\mu\nu}\,T^\lambda_{\phantom{\lambda}\lambda}~,
\]
	and similarly for 3-tensor objects.
	The important feature of this splitting is that each of the terms 
	has a specific spin defined by the number of indices --- all lower-spin
	contributions have been explicitly subtracted. 
	In other words, a tensor is irreducible if, using $ g^{\mu\nu} $ and
	$ \epsilon^{\mu\nu\rho\sigma} $, one cannot form a structure of a lower
	spin. 
	In the example above we get an antisymmetric tensor, a symmetric tensor and a 
	scalar trace.

	However, this break-up is not the ultimate list: one also has to exclude
	those operators which can be reduced on the equations of motion:
%%
%% the EOM
\begin{equation}
\label{EOM}
	\sslashed{\mathcal{D}} \psi = - i\, m \psi~.
\end{equation}
	For some of them, Lorentz-irreducibility automatically means irreducibility
	on the EOM, for others --- not.

	In constructing objects of dimension 5, we are limited to only considering
	Dirac fermion bilinears --- the next possible combination would be 	
	fermion quadrilinears which already belong to dimension 6 and higher.
	That is, it is enough to consider all different kinds of bilinears
\begin{equation}
\label{generic_term}
	\overline{\psi}\, \gamma^{\mu} \gamma{^\nu} ..\, 
	\mathcal{D}^{\rho} \mathcal{D}^{\sigma}.. \,\psi~,
\end{equation}
	where $ \mathcal{D}^\mu $ is a gauge derivative in general carrying respective
	gauge fields pertinent to $ \psi $.
	One more simplification comes from the observation that one cannot construct
	too many combinations of $ \gamma $-matrices. 
	More importantly, the Standard Model is known to have \emph{chirality}, 
	i.e. left- and righthand species couple differently to the gauge fields.
	That means that one cannot form a gauge invariant fermion bilinear from
	a left and a right particle (without involving Higgs, which we agreed 
	not to) --- in the chiral basis they have to be either both right or 
	both left. 
	One could of course argue that at lower energies, where only electromagnetism
	and strong interactions exist, the respective charges are the same
	for the right and left species, and thus one could try to compose bilinears
	formed from species of different chiralities. 
	However, we emphasize that the nature of Lorentz Violation comes to 
	rule at some high energy scales, far beyond the Electroweak scale, and
	therefore we assume the LV terms to be gauge invariant under all
	(at least by now) known gauge symmetries, including EW symmetry which
	pronounces the difference between the right- and lefthanded species. 
	This leads us to an observation that in a generic basis, 
	in between the fermions in \eqref{generic_term}, one could only place an 
	odd number of $ \gamma $-matrices. 
	Indeed, an object like $ \overline{\psi}_L \gamma^\mu \gamma^\nu \psi_L $,
	with an even number of $ \gamma $-matrices,
	vanishes identically due to chirality projectors.
	This way, as three $ \gamma $-matrices can always be reduced, 
	we only need to consider the bilinears with one $ \gamma $-matrix inserted:
\[
	\overline{\psi}\, 
	\gamma^{\mu}\, \mathcal{D}^{\rho} \mathcal{D}^{\sigma}.. \,\psi~,
	~~~
	\overline{\psi}\, 
	\gamma^{\mu}\gamma^5\, 
	\mathcal{D}^{\rho} \mathcal{D}^{\sigma}.. \,\psi~.
\]
	Expanding the Lorentz structures into irreducible components we arrive
	at the following list of the operators, to-be-irreducibles:
%%
%% list of Lorentz-irreducible operators 
\begin{eqnarray}
\label{irred_list}
% first line -- vectors
\nonumber
	&&
	\overline{\psi} \gamma^\mu \mathcal{D}^2 [\gamma^5] \psi~,
	\qquad
	\overline{\psi} \left\{ \mathcal{D}^\mu \sslashed{\mathcal{D}} \right \}
		[\gamma^5] \psi~,
	\qquad
	\overline{\psi} \gamma_\lambda F^{\mu\lambda} [\gamma^5] \psi~,
	\qquad
	\overline{\psi} \gamma_\lambda \widetilde{F}^{\mu\lambda} [\gamma^5] \psi~,
	\\
% second line -- 3-tensors
	&&
	\overline{\psi} \gamma^{(\mu} \mathcal{D}^\nu \mathcal{D}^{\rho)}
		[\gamma^5] \psi ~-~ \langle\text{traces}\rangle~,
	\qquad
	\overline{\psi} \gamma^{(\mu} F^{\rho)\nu} 
		[\gamma^5] \psi ~-~ \langle\text{traces}\rangle~,
	\\
% third line -- 3-tensors
\nonumber
	&&
	\overline{\psi} \left( \mathcal{D}^{(\mu} \mathcal{D}^{\nu)} \gamma^\rho
		~-~ \mathcal{D}^\rho \mathcal{D}^{(\mu}\gamma^{\nu)} \right )
		[\gamma^5] \psi ~-~ \langle\text{traces}\rangle~,
\end{eqnarray}	
	where $ [\gamma^5] $ signifies that one can also insert
	a $ \gamma^5 $ in each term, thus effectively doubling the above list.
	Also, $ \langle\text{traces}\rangle $ means subtraction of all possible
	traces (including ones obtained with $ \epsilon^{\mu\nu\rho\sigma} $)
	from a given structure. 
	We omit from now on the label $ \langle\text{traces}\rangle $,
	assuming all tensor structures to be irreducible.

	The next stage is to remove from the above list all those terms
	which can be reduced on the equations of motion \eqref{EOM}.
	It turns out that most of the structures in \eqref{irred_list}
	die away this way.
	We therefore arrive at a generic Lorentz-violating lagrangian
%%
%% Generic LV lagrangian
\begin{eqnarray}
\label{generic_LV}
% zeroth line
\nonumber
\lefteqn{
	\mathcal{L}_{\rm LV} =
	}
	\\
% first line
\nonumber
	&&
	c^\mu \cdot \overline{\psi} \gamma^\lambda F_{\lambda\mu} \psi
	~+~
	d^\mu \cdot \overline{\psi} \gamma^\lambda F_{\lambda\mu} \gamma^5 \psi
	~+~
	\widetilde{c}^\mu \cdot \overline{\psi} \gamma^\lambda 
	\widetilde{F}_{\lambda\mu} \psi
	~+~
	\widetilde{d}^\mu \cdot \overline{\psi} \gamma^\lambda 
	\widetilde{F}_{\lambda\mu} \gamma^5 \psi
	~+~
	\\
% second line
	& + &
	C^{\mu\nu\rho}_1 \cdot \overline{\psi} 
	\gamma_{(\mu} \mathcal{D}_\nu \mathcal{D}_{\rho)} \psi
	~+~
	C^{\mu\nu\rho}_2 \cdot \overline{\psi} 
	\gamma_{(\mu} \mathcal{D}_\nu \mathcal{D}_{\rho)} \gamma^5 \psi 
	~+~
	\\
% third line
\nonumber
	& + &
	D^{\mu\nu\rho}_1 \cdot \overline{\psi}
	\gamma_{(\mu} F_{\rho)\nu} \psi
	~+~
	D^{\mu\nu\rho}_2 \cdot \overline{\psi}
	\gamma_{(\mu} F_{\rho)\nu} \gamma^5 \psi~.
\end{eqnarray}
	Looking more attentively, by noting that 
\begin{equation}
\label{fstrength}
 	F_{\mu\nu} = - \, i e\, [\mathcal{D}_{\mu} \mathcal{D}_{\nu}]~,
\end{equation}
	one can see that the first two terms in \eqref{generic_LV}
	actually vanish. 
	However, that is true for purely electromagnetically interacting
	particles only --- i.e., for leptons --- because in general
	a commutator of two $ \mathcal{D}_\mu $'s yields an additional
	gluon fieldstrength $ G_{\mu\nu} $.
	We will consider this option later, for now leaving the two terms
	where they are.

	We can see, that \eqref{generic_LV} includes different tensor structures:
	some of which include absolutely symmetric tensors $ C^{\mu\nu\rho} $, other
	--- partially symmetric tensors $ D^{\mu\nu\rho} $. 
	As mentioned in \cite{MP:}, $ C^{\mu\nu\rho} $ must be irreducible tensors
	because reducible parts of them would have induced quadratic divergencies.
	Here we observe that the requirement of Lorentz-irreducibility of the
	operators $ C^{\mu\nu\rho} $ comes also as the consequence of the irreducibility
	on the EOM: firstly, if $ C^{\mu\nu\rho} $ were reducible, its reducible part
	will correspond to a lower-spin operator which thus should have been 
	written out separately (c.f. the first line in \eqref{generic_LV}); 
	secondly, those lower-spin parts actually are
	reducible on the EOM --- and \emph{thus} have to be subtracted out.

	Before generalizing the lagrangian \eqref{generic_LV} to the Standard
	Model, we first turn to the low-energy effects induced by it
	(generalization of which to the Standard Model case are straightforward).
	For most of the contributions in \eqref{generic_LV} it is instructive to
	find out their corrections to the nonrelativistic hamiltonian.
	The terms $ C_{1,2}^{\mu\nu\rho} $ however, only affect the 
	dispersion relations of the particle 
\cite{MP:}.

	In Table \ref{CPT_table} below we present the CPT properties of the 
	operators discussed above, together with the nonrelativistic 
	hamiltonian contributions they make.  
	For showing the $ P $- and $ T $- transformation properties, we have assumed
	the external vector and tensor backgrounds to be timelike, and taken the zeroth
	components of them --- except for the operators $ D^{\mu\nu\rho} $, for 
	which the components $ D^{000} $ do not exist; in the latter case we 
	have chosen purely vector components $ D^{ijk} $ as the representatives.
%%
%% CPT-properties and H_eff
\begin{table}[tb]
\label{CPT_table}
\begin{equation*}
\begin{array}{l||ccc|c|c}
\hline
     &  C  &  P  &  T  &  CPT  & 
	\mathcal{H}_{\rm eff} ~/~ {\rm Dispersion\ relations} \\
\hline
c^\mu \cdot \overline{\psi} \gamma^\lambda F_{\lambda\mu} \psi &
	~~+~~  &  ~~+~~  &  ~~-~~  &  ~~-~~    & 
	-\, \frac{1}{2m} \left\{ \vec{\pi}\, , \,
				 c_0 \vec{E} ~+~ 
		\left[\vec{c} \times \vec{B} \right] \right\} \\

d^\mu \cdot \overline{\psi} \gamma^\lambda F_{\lambda\mu} \gamma^5 \psi &
	-  &  -  &  -  &  -    & 
	d_0 (\vec{E}\cdot\vec{\sigma}) ~+~ 
	\left[ \vec{d} \times \vec{B} \right] \cdot \vec{\sigma}
	~-~ \frac{1}{2m} \left\{ \vec{\pi}\cdot\vec{\sigma}\, , \,
			\vec{d} \cdot \vec{E} \right\}\\

\widetilde{c}^\mu \cdot \overline{\psi} \gamma^\lambda 
	\widetilde{F}_{\lambda\mu} \psi &
	+  &  -  &  +  &  -    & 
	- \frac{1}{2m} \left\{ \vec{\pi} \, , \,
		\widetilde{c}_0 \vec{B} ~-~
		\left[ \vec{\widetilde{c}} \times \vec{E} \right ] \right\} \\

\widetilde{d}^\mu \cdot \overline{\psi} \gamma^\lambda 
	\widetilde{F}_{\lambda\mu} \gamma^5 \psi &
	-  &  +  &  +  &  -    & 
	\widetilde{d}_0 (\vec{B} \cdot \vec{\sigma}) ~-~
	\left[ \vec{\widetilde{d}} \times \vec{E} \right] \cdot \vec{\sigma} ~-~
	\frac{1}{2m} \left\{ \vec{\pi}\cdot\vec{\sigma}\, , \,
			\vec{\widetilde{d\,}} \cdot \vec{B} \right\}\\

C^{\mu\nu\rho}_1 \cdot \overline{\psi} 
	\gamma_{(\mu} \mathcal{D}_\nu \mathcal{D}_{\rho)} \psi &
	-  &  +  &  +  &  -    & 
	\Bigl\{\, E^2 ~-~ \vec{p}^2 ~-~ m^2 ~-~ \qquad\qquad\qquad\qquad \\

C^{\mu\nu\rho}_2 \cdot \overline{\psi} 
     \gamma_{(\mu} \mathcal{D}_\nu \mathcal{D}_{\rho)} \gamma^5 \psi &
	+  &  -  &  +  &  -    & 
	~-~ 12\, \left [ C_1^{\mu\nu\rho} ~+~ C_2^{\mu\nu\rho}\gamma^5 \right ]\;
	p_\mu\, p_\nu\, p_\rho \, \Bigr \} \, \psi = 0 \\

D^{\mu\nu\rho}_1 \cdot \overline{\psi}
	\gamma_{(\mu} F_{\rho)\nu} \psi &
	+  &  -^*  &  +^*  &  -    & 
         -\,\frac{1}{2m}
	\left\{ \vec{\pi}\,,\, D_1^{\mu)\nu(\rho} F_{\rho\nu} \right\}
	\\

D^{\mu\nu\rho}_2 \cdot \overline{\psi}
	\gamma_{(\mu} F_{\rho)\nu} \gamma^5 \psi &
	-  &  +^*  &  +^*  &  -    & 
	D_2^{k)\nu(\rho} F_{\rho\nu} \cdot \sigma^k
	\\
\hline
\end{array}	
\end{equation*}
\caption{CPT-properties of the dimension five LV operators and their 
	contributions to the effective non-relativistic hamiltonian.
	To exhibit the P,T-properties we have chosen the zeroth components
	of the LV tensors.
	An asterisk marks the signs where we chose the vector components
	of the corresponding background tensors. 
	Gauge covariantized momentum is denoted as 
	$ \vec{\pi} = \vec{p} + e\vec{A} $.}
\end{table}

	From this table we can clearly see which operators present the most
	phenomenological interest. 
	Evidently, the operators $ c^\mu $ and $ \widetilde{c}^\mu $ make
	contributions which are suppressed by the mass of the (nonrelativistic)
	particle, and so is the contribution of $ D_1^{\mu\nu\rho} $.
	Therefore we concentrate on the operators $ d^\mu $, $ \widetilde{d}^\mu $
	and $ D_2^{\mu\nu\rho} $.
	We will consider them in the reverse order.
	But first we note that the only effect of the operators $ C^{\mu\nu\rho} $
	is to modify the dispersion relations. 
	This has been studied in detail in \cite{MP:}.
	These operators do not induce any coupling of the particle's spin to 
	the electromagnetic field. 
	In fact, they do not couple the electromagnetic field to anything ---
	the expansion into irreducibles has done the job to sweep away all
	electromagnetic fieldstrength contributions from the $ C^{\mu\nu\rho} $
	operators spreading them between the other operators.
	
	We can also see that the operator $ D_2^{\mu\nu\rho} $
	couples particle's spin to the electromagnetic field, and thus we might
	suspect that it could also induce an electric or magnetic dipole moment.
	In order to see whether this takes place or not, we decompose $ F_{\rho\nu} $
	into electric and magnetic fields, and then expand the resulting tensor
	structures into irreducibles with respect to the 3-dimensional indices.
	It appears that the irreducibility of $ D_2^{\mu\nu\rho} $ under 
	4-dimensional Lorentz transformations has important consequences and 
	puts some constraints onto the 3-dimensional structures.
	We find the following couplings induced by the $ D_2^{\mu\nu\rho} $ 
	operator:
%%
%% H_eff of D_2
\begin{equation}
\label{HeffD2}
	\mathcal{H}_{\rm eff} ~\supset~
		\vec{L} \cdot \left[ \vec{E} \times \vec{\sigma} \right]
	~+~
	L^{ik}\, E^i \sigma^k 
	~+~
	K^{ik}\, B^i \sigma^k~,
\end{equation}
	where $ \vec{L}^i = - \epsilon_{ikl} D_2^{0kl} $, and $ L^{ik} $
	and $ K^{ik} $ are traceless tensors:
%%
%% definition of L_ik and K_ik
\[
	L^{ik} ~=~ D_2^{i[0k]} ~+~ D_2^{k[0i]}~,
	\quad
	K^{ik} ~=~ -\, \epsilon^{ml(i} D_2^{k)ml}~,
\]
	tracelessness of which follows
	from the requirement of the irreducibility of $ D_2^{\mu\nu\rho} $.
	Precisely because they are traceless, the couplings \eqref{HeffD2} do
	not contain electric or magnetic dipole moments.

	The operator $ \widetilde{d} $ clearly carries a signature of the 
	magnetic moment of a particle --- in fact, a CPT-violating magnetic moment
	--- which can be restricted by the known limits on the difference of the 
	magnetic moments of the electron and positron \cite{SQED}.

	The operator $ d^\mu $ presents the topmost interest to us.
	It has a signature of a CP-conserving and simultaneously CPT-violating
	Electric Dipole Moment. 
	Generically, this EDM adds up to the CPT-conserving dipole moment
	$ \vec{E} \cdot \vec{\sigma} $, and so 
	in order to restrict the former one, one needs to be able to observationally
	tell one from the other. 
	As is well-known 
\cite{Ginges}, in heavy paramagnetic atoms, the EDM of the 
	electron can be significantly increased. 
	This enhancement could in general be different for CPT-conserving and
	CPT-violating EDMs, that is it could scale with Z of the atoms 
	differently for the two cases.
	In this case one could hope that, upon detecting a non-zero effect,
	by measuring and comparing EDMs of paramagnetic atoms with 
	different Z's, one would be
	able to distinguish between the two kinds of EDMs.

	It turns out however, that for the case of the low-energy electron,
	the operator $ d^\mu $ (along with $ c^\mu $) vanishes due to the
	observation \eqref{fstrength}.
	In other words, we have not induced the EDM for the electron, and 
	there is nothing to be detected.
	Nevertheless, the situation is different for the quark sector.

%%%%%%%%%%%%%%%%%%%%%%%%%%%%%%%%%%%%%%%%%%%%%%
%%%%
%%%               The LV in the Standard Model
%%%%
%%%%%%%%%%%%%%%%%%%%%%%%%%%%%%%%%%%%%%%%%%%%%%
\section{The LV in the Standard Model}

	So far we have been considering only U(1) gauge interactions, that is
	to say, electromagnetism. 
	Now we generalize these results to the GeV-scale Standard Model. 
	Generically, we would have to enumerate the structures with all possible field
	content: leptons, quarks, gluons, Higgs, weak gauge bosons. 
	The latter two, however, are not relevant for the problems of laboratory
	experiments, which are being conducted at low energies
	(in other words, despite that we still could list all LV operators which 
	include Higgs and gauge bosons, at low energies they are heavy and can 
	be integrated out, effectively leaving us with LV operators of 
	higher dimensions).
	Therefore, here we can limit ourselves with only the lepton and 
	quark sectors,
	which in their turn only interact electromagnetically and strongly.
	That is, our gauge field content now includes gluons, matter content ---
	$ u, d, s $-quarks and leptons (basically, the electron, as the
	muon is less phenomenologically viable). 
	
	Now indeed we can get rid of the first two terms ($ c^\mu $ and
	$ d^\mu $) in \eqref{generic_LV}, as they proclaimedly vanish
	in the electromagnetic sector, but 
	they still are present in the quark sector. 
	All the other difference of the Standard Model with respect to
	U(1) theory is that now we have to include the gluon fieldstrength
	in the quark sector 
	in all places where we have included the electromagnetic fieldstrength.

	Denoting by $ \psi $ a lepton and 
	by $ q $ a quark ($ q = u, d, s $), 
	a GeV-scale dimension 5 Lorentz Violating Standard Model takes the
	form:
%%
%% Leptonic LV operators
\begin{eqnarray}
\label{LV_SM_EM}
% first line
\nonumber
\lefteqn{
	\mathcal{L}^{\rm EM}_{\rm LV} =
	}
	\\
% second line
\nonumber
	&&
	\widetilde{c}^\mu \cdot \overline{\psi} \gamma^\lambda 
	\widetilde{F}_{\lambda\mu} \psi
	~+~
	\widetilde{d}^\mu \cdot \overline{\psi} \gamma^\lambda 
	\widetilde{F}_{\lambda\mu} \gamma^5 \psi
	~+~
	\\
% third line
\nonumber
	& + &
	C_{\mu\nu\rho} \cdot \overline{\psi} 
	\gamma^{(\mu} \mathcal{D}^\nu \mathcal{D}^{\rho)} \psi
	~+~
	\widetilde{C}_{\mu\nu\rho} \cdot \overline{\psi} 
	\gamma^{(\mu} \mathcal{D}^\nu \mathcal{D}^{\rho)} \gamma^5 \psi 
	~+~
	\\
% fourth line
\nonumber
	& + &
	D_{\mu\nu\rho} \cdot \overline{\psi}
	\gamma^{(\mu} F^{\rho)\nu} \psi
	~+~
	\widetilde{D}_{\mu\nu\rho} \cdot \overline{\psi}
	\gamma^{(\mu} F^{\rho)\nu} \gamma^5 \psi~,
	\\
\end{eqnarray}
	in the leptonic sector and
%%
%% QCD LV operators
\begin{eqnarray}
\label{LV_SM_QCD}
% fifth -- QCD lagrangian
\lefteqn{
	\mathcal{L}^{\rm QCD}_{\rm LV} =
	}
	\\
% sixth line
\nonumber
	&&
	c_q^\mu \cdot \overline{q} \gamma^\lambda F_{\lambda\mu} q
	~+~
	d_q^\mu \cdot \overline{q} \gamma^\lambda F_{\lambda\mu} \gamma^5 q
	~+~
	\widetilde{c}_q^\mu \cdot \overline{q} \gamma^\lambda 
	\widetilde{F}_{\lambda\mu} q
	~+~
	\widetilde{d}_q^\mu \cdot \overline{q} \gamma^\lambda 
	\widetilde{F}_{\lambda\mu} \gamma^5 q
	~+~
	\\
% seventh line
\nonumber
	& + &
	c_g^\mu \cdot \overline{q} \gamma^\lambda G_{\lambda\mu} q
	~+~
	d_g^\mu \cdot \overline{q} \gamma^\lambda G_{\lambda\mu} \gamma^5 q
	~+~
	\widetilde{c}_g^\mu \cdot \overline{q} \gamma^\lambda 
	\widetilde{G}_{\lambda\mu} q
	~+~
	\widetilde{d}_g^\mu \cdot \overline{q} \gamma^\lambda 
	\widetilde{G}_{\lambda\mu} \gamma^5 q
	~+~
	\\
% eighth line
\nonumber
	& + &
	C^{\mu\nu\rho}_q \cdot \overline{q} 
	\gamma_{(\mu} \mathcal{D}_\nu \mathcal{D}_{\rho)} q
	~+~
	\widetilde{C}^{\mu\nu\rho}_q \cdot \overline{q} 
	\gamma_{(\mu} \mathcal{D}_\nu \mathcal{D}_{\rho)} \gamma^5 q 
	~+~
	\\
% ninth line
\nonumber
	& + &
	D^{\mu\nu\rho}_q \cdot \overline{q}
	\gamma_{(\mu} F_{\rho)\nu} q
	~+~
	\widetilde{D}^{\mu\nu\rho}_q \cdot \overline{q}
	\gamma_{(\mu} F_{\rho)\nu} \gamma^5 q
	~+~
	\\
% tenth line
\nonumber
	& + &
	D^{\mu\nu\rho}_g \cdot \overline{q}
	\gamma_{(\mu} G_{\rho)\nu} q
	~+~
	\widetilde{D}^{\mu\nu\rho}_g \cdot \overline{q}
	\gamma_{(\mu} G_{\rho)\nu} \gamma^5 q~
\end{eqnarray}
	in the QCD sector.

	The effects of the newly introduced operators are similar
	to those listed in the Table \ref{CPT_table}. 
	In particular, $ d_g^\mu $, $ \widetilde{d}_g^\mu $ lead to
	chromoelectric and chromomagnetic dipole moments correspondingly. 
	However, we can again pull up the relation \eqref{fstrength}
	and observe that even though it does not lead to a total vanishing
	of the $ c^\mu $ and $ d^\mu $ parameters for the quarks, we
	still can use it to trade the chromomagnetic and chromoelectric
	operators for the electric and magnetic dipole moment ones. 
	Therefore, without loss of generality we can put
\[
	c^\mu_g ~=~ d^\mu_g ~=~ 0~.
\]

	Now the operator $ d_q^\mu $, obviously, results in the
	EDM of quarks. 
	That inevitably leads to nucleons' having an EDM.
	One can estimate the contribution to the nuclear EDM
	by averaging over the nucleon the spin part of the operator
	$ d_q^\mu $:
\begin{equation}
\label{nucleon_EDM}
	d_N ~~\supset~~ 
		d^0_u\, \langle N |\, \overline{u} \gamma^i \gamma^5 u \,| N \rangle 
		~~+~~
		d^0_d\, \langle N |\, \overline{d} \gamma^i \gamma^5 d \,| N \rangle 
		~~+~~
		d^0_s\, \langle N |\, \overline{s} \gamma^i \gamma^5 s \,| N \rangle 
	~.
\end{equation}
	Then, beneficially, one can use the nuclear spin decomposition
	\cite{Ellis94}, in order to limit the corresponding
	LV parameters $ d_u^\mu $, $ d_d^\mu $ and $ d_s^\mu $.
	As per \cite{Ellis94},
\[
	\langle p |\, \overline{u} \gamma^\mu \gamma^5 u \,| p \rangle
	= 0.83~,~~
	\langle p |\, \overline{d} \gamma^\mu \gamma^5 d \,| p \rangle
	= -0.43~,~~
	\langle p |\, \overline{s} \gamma^\mu \gamma^5 s \,| p \rangle
	= -0.10
	~.
\]
	For the neutron, one correspondingly has to change 
$ u \leftrightarrow d $.
	This can be combined with the estimate from \cite{Pospelov:2005pr} for
	the neutron EDM $ d_n $, to obtain:
\begin{equation}
\label{combined_limits}
	\left |\; -\,0.43 \cdot d^0_u  ~~+~~ 0.83 \cdot d^0_d ~~-~~ 0.10 \cdot d^0_s 
	 \; \right|
	~~<~~ 
	6 \times 10^{-26} e\; {\rm cm}~.
\end{equation}

	EDM of the nucleons, in their turn, induce EDM of the nuclei. 
	For diamagnetic atoms, there is a very strong limit 
	\cite{Ginges}: 
$ d({}^{199}{\rm Hg}) ~~\lesssim~~ 1 \times 10^{-28}\, e~{\rm cm} $,
	which can be translated into an estimate quite competitive with
	\eqref{combined_limits}.

	By assuming that the LV backgrounds for the different quark species
	are not so much well-correlated as to produce this suppression, we infer
	the bounds $ \left| d^0_{u,d,s} \right| \lesssim 10^{-26} e\,{\rm cm} $.
	In general, other quarks also contribute to \eqref{nucleon_EDM}, but
	it is hard to make estimates for them without knowing their weight
	in the nuclear spin decomposition.

	As we assume the LV vector backgrounds to be time-like, and presumably
	associated with the cosmic microwave background frame, we can argue
	that their space components must be smaller than their time components,
	and therefore give a collective estimate
\[
	| d^\mu_u |~,~~ | d^\mu_d |~,~~ | d^\mu_s | ~~<~~ 10^{-26} e\; {\rm cm}~.
\]



\newpage
%%%%%%%%%%%%%%%%%%%%%%%%%%%%%%%%%%%%%%%%%%%%%%
%%%%
%%%               Conclusion
%%%%
%%%%%%%%%%%%%%%%%%%%%%%%%%%%%%%%%%%%%%%%%%%%%%
\section{Epilog}
	
	We have introduced Lorentz Violation in terms of dimension five operators
	coupled to external tensorial backgrounds into the matter sector of the
	Standard Model, which serve to imprint the signatures of the New Physics. 
	All possible tensor structures admitted by gauge invariance have been
	examined and classified with respect to their C-, P-, T- properties 
	and low-energy implications.
	Among them are the LV operators pertinent to the Myers-Pospelov 
	Electrodynamics \cite{MP:}, the most important low-energy manifestation of 
	which
	is the $ ({\rm Energy})^3 $ modification of the dispersion relations. 
	The requirements of irreducibility under Lorentz transformations and 
	of irreducibility on the Equations of Motion have been shown to
	sweep out a number of candidate operators.

	We have seen that one of the most important low-energy effects of the
	LV at the mass dimension five level is to induce Electric Dipole Moments
	for different particle sectors. 
	The resulting EDM possesses the common property of dimension five operators
	--- CPT-noninvariance.
	However, it eludes direct experimental probes targeted for the EDM
	of the electron, as the structure of the original LV operator causes it
	to vanish below the EW scale in leptonic sector.
	Nevertheless, due to the additional color SU(3) symmetry, the EDM-inducing 
	structure is persistent in the quark sector.
	By observing that quark EDM induces an EDM in nucleons, we have used the 
	nuclear spin decomposition and put constraints on the corresponding 
	quark sector LV parameters via one of the best to-date limits on the
	nucleon EDM \cite{Pospelov:2005pr}.

%%%%%%%%%%%%%%%%%%%%%%%%%%%%%%%%%%%%%%%%%%%%%%%%%%%%%%%%%%%%%%%%%%%
%%%%%%%%%%%%%%%%%%%%%%%%%%%%%%%%%%%%%%%%%%%%%%%%%%%%%%%%%%%%%%%%%%%
%%%%                                                           %%%%
%%%                      Bibliography                           %%%
%%%%                                                           %%%%
%%%%%%%%%%%%%%%%%%%%%%%%%%%%%%%%%%%%%%%%%%%%%%%%%%%%%%%%%%%%%%%%%%%
%%%%%%%%%%%%%%%%%%%%%%%%%%%%%%%%%%%%%%%%%%%%%%%%%%%%%%%%%%%%%%%%%%%

\bibliographystyle{apsrev}
\bibliography{edm}

\end{document}
